\documentclass[12pt,numbers=noenddot,parskip,bibliography=totocnumbered,listof=totocnumbered]{scrreprt}
\usepackage[paper=a4paper,left=30mm,right=25mm,top=30mm,bottom=30mm]{geometry}
\usepackage[utf8]{inputenc}
\usepackage[hyphens]{url}
\usepackage[english]{babel}
\usepackage{tocstyle}
\usetocstyle{allwithdot}
\usepackage{lmodern}
\usepackage[sort&compress]{natbib}
\usepackage{graphicx} 
\usepackage{subcaption}
\usepackage{rotating}
\usepackage{transparent}
\usepackage{multirow}
\usepackage{tabularx}
\usepackage{pifont}
\usepackage{setspace}
\usepackage{amsmath}
\usepackage[automark]{scrlayer-scrpage}

% Stil der Seiten
\pagestyle{scrheadings}
\clearscrheadfoot

%Abstand der Fussnoten
\deffootnote{1em}{1em}{\textsuperscript{\thefootnotemark\ }}

%Zeilenabstand
\onehalfspacing

%Regeln, bis zu welcher Tiefe Überschriften angezeigt werden sollen (Anzeige der Überschriften im Verzeichnis / Anzeige der Nummerierung)
\setcounter{tocdepth}{3}
\setcounter{secnumdepth}{3}

%Kopfzeile (Kapitel / Standard)
\ohead{\transparent{0.5}\headmark\transparent{0.5}\transparent{1}}

%Fußzeile (Kapitel / Standard)
\ofoot[\rlap{\hspace{.5cm}\rule[-0.7ex]{.8pt}{\baselineskip}\hspace{.5cm}\pagemark}]{\rlap{\hspace{.5cm}\rule[-0.7ex]{.8pt}{\baselineskip}\hspace{.5cm}\pagemark}}

%Kapitel in Kopfzeile ohne Zahl
\renewcommand*{\chaptermarkformat}{}

%Schriftarten
\addtokomafont{pagenumber}{\sffamily \upshape}
\addtokomafont{pageheadfoot}{\sffamily \upshape}\usepackage[defaultfam,light,tabular,lining]{montserrat}
\usepackage[T1]{fontenc}
\renewcommand*\oldstylenums[1]{{\fontfamily{Montserrat-TOsF}\selectfont #1}}

\begin{document}
	
% TItelseite
\begin{titlepage}
\null
\vfill
% The Collective Guide: A machine for collective recommendations and its desire for disclosure
% Honestly transforms experiences in a collective into public recommendations
% The Honest Machine: Transforms experiences in a collective into honest recommendations
% The bold attempt of transforming experiences in a collective into  recommendations
% The Honest Machine: The attempt of converting experiences in a collective into honest recommendations

% Felix: Sollte beinhalten...
% Recommendation
% Virtuell vernetzt bzw. wohnt im Netz
% Keine Negation
% Freizeitgestaltung verbessern

\Huge\textsf{\textbf{Free your free time
\vspace{0.5em}}}\\
\LARGE\textsf{ Assessment of a recommendation machine to increase 
%one's quota of 
the amount of
spare time spent collectively 
%and in person 
} \vspace{1.5em}\\
\Large\textsf{Jan-Hendrik Wolf}
\vfill
\vfill
\vfill
\small{Thesis submitted to the University Bremen in partial fulfillment of the requirements for a M.Sc. degree in Digital Media\\
Bremen, 30. Juli 2017}
\end{titlepage}

% Anfang des Bodies
\pagenumbering{roman}

%Inhaltsverzeichnis
\tableofcontents

\chapter*{Abstract}

\chapter{Introduction}
\pagenumbering{arabic}

% Argumente:
% Große Auswahl
% Oft solitäre Dinge
% Nicht Facebook weil dies die Online-Zeit erhöhen will

% Abgrenzen zu: 
% Wirkliche Freizeit, nicht Arbeit
% Nicht ungeplante Zeit
% Die Freizeitfüllung nicht digital sondern offline stattfinden lassen

Gone are the days of leisure and self-determined free time. Gone are the days of accepting boredom and time of waiting. The influence of media changed the way we organize our free time.  \\ 
Social networks like Facebook or search engines like Google opened up enough opportunities to choose from and made it unreasonable to feel boredom and vacancy, which was an acceptable part of our free time thirty years ago. Not staying in touch over a certain timeframe is not taken for granted anymore due to the growing number of channels in communication. We stress ourselves, trying to fill any possible gap in our free time. Any intervening period during a bus ride, in the waiting room or alone at home is filled with a constantly growing abundance of minor activities. Maintain your friendships, social web profiles and all manner of incoming information.The digitalization made all this possible and the smartphone is the most popular toolbox to pursue these habits. \\ %quelle
This development is continued by examples like shopping malls become larger, the number of sports grows and even ordinary commercial products are sold as a new experience. This confusing number of opportunities make us feel afraid to miss a thing. We are constantly trying to transform interims into productive free time, anyhow refuse ourselves to the unexpected. According to a series of studies \citep[p. 14]{freizeitmonitor2016}, we are constantly increase our personal stress level this way. The artificial impression is build up that time is running short. 

This phenomenon has impact on the way we spend our free time. The winning activities for the last five years had been activities you are doing alone, whereas social activities had been the most notably losers. It seems evident that it takes more time to meet and spend time with each other personally, than staying in contact via messengers and social media. However, why do we not use the advantages of computer technology in a way that motivates people to preferably spend their free time in a context of social activities again? Is it possible that an application, the embodiment of digitalization in the current state, can invert the trend towards a free time with social activities in focus?

The thesis will call attention to the constant change of one's free time under decades of digital influence. It will investigate to what extend the ubiquitous accessibility of everyones attention certainly played its part to this development and proposes a possible solution to invert the trend. A puristic application that does not focus on chat components, self-portraits and its time-consuming relatives. The aim is to conceptualize and implement an application that helps to organize one's free time with a small timestamp online and its constant motivation, in contrary to existing solutions like Facebook, to make people meet each other in person again rather than digitally. The thesis will end with an evaluation of application users to validate whether it is a rather drastic and humorously ironic idea to come up with another digital application or a step forward to one's organization free time with more social activities.

% single activity does not last  more than two hours. 
% most notably surfing the web, listening to music and go to the gym.
% Fülle an Ideen für Freizeitaktivitäten und Interessen.
% Menschen kennen meist nur einen kleinen Bereich mangels des Wissens der anderen Bereiche.
% Ratgeber sind nicht individuell auf den Einzelnen abgestimmt, parteiisch und konzentrieren sich auf spezielle Ereignisse.
% Seit XXXX gibt es Empfehlungsalgorithmen, werden zunehmend beliebter und es gibt eine Fülle an nützlichen Anwendungsbeispielen: Diese Algorithmen könnte man nutzen, um Nutzern bei der Filterung zu helfen und die Ideen und Interessen an einem Ort zu speichern.
%Bestehende, automatisierte Ratgeber (z.B. Facebook) geben undurchsichtige Vorschläge - es wird keine Auskunft über über deren Generierung gegeben. 
%Wieviele Quellen bezieht die Empfehlung bspw. ein? Mystifizierte Vorschläge.

\chapter{The Evolution Of Free Time}

The following chapter gives a general definition of free time that is used in the context of this thesis. The second section provides a historical overview of the development of free time and summarizes key changes towards a society of time pressures. In order to compare the machine to alternatives, other contemporary organization tools are discussed at the end of this chapter.

\section{Definition}

There is an ongoing progress of defining free time in leisure studies. Several definitions of free time and leisure time have been constituted in multiple ways. German leisure studies distinguish between positive and negative definitions of leisure. \newline
Originating in the protestant work ethic \citep[p.27]{weber2006} and industrialization, negative definitions of leisure describe the time off from non-eligible activities, including employment, transit time, hygiene and eating \citep[p.137]{prahl2002}. Unfortunately these definitions partially exclude individuals, who are unemployed, elderly or teenagers. \citeauthor{scheuch1972} tries to overcome this major weakness of negative definitions and suggests to set the type of activity in relation to the functional role of the individual in society. Leisure time is not meant to be a quantitative time unit anymore, but is rather defined as a phenomenon of its own \citep[p.31]{scheuch1972}. With \citeauthor{scheuch1972} the development of new definitions of leisure started.\newline
Positive definitions of leisure focus on the fact, that an activity ``has been chosen primarily for its own sake, for the experience itself'' \citep[p.15]{freysinger2000}. Quality becomes the predominant feature of leisure\footnote{In most cases, it is easy to determine, whether someone enjoys an activity and feels satisfied. Consequently positive definitions are a good way to distinguish between leisure and non-leisure time. But does an individual spend leisure time by doing sports in the gym to loose weight? Does a pop band, who plays for a hobby, have leisure time at a party when asked to play, although they rather would like to enjoy the party without performing? Sometimes it is not possible to distinguish, whether someone is spending leisure or non-leisure time. Only the persons themselves can tell.}  and is no longer bound to soley be classified as time. Positive definitions of leisure are the most common way of defining leisure internationally.

In international leisure studies the definitions of free time are comparable with negative definitions of leisure mentioned above. The American sociologist \citeauthor{stebbins2007} describes free time as the ``time away from unpleasant obligation'' \cite[p.4]{stebbins2007}, as a quasi subtraction of the unpleasant from one's totally available time. \citeauthor{stebbins2007} defintion shows little variation to negative definitions of leisure. In accordance with positive definitions of leisure \citeauthor{stebbins2007} defines leisure as an ``uncoerced  activity engaged in during free time, which people want to do and, in either a satisfying or a fulfilling way (or both), use their abilities and resources to succeed at this''. 

There are numerous definitions of free time and leisure and the entire field is complex. In this thesis My discussion in this thesis will use the definitions constituted by \citeauthor{stebbins2007} as his statements are well applicable to positive and negative definitions of leisure. In summary, \textit{free time} in this thesis is meant to be the time away from eligible activities, whereas \textit{leisure time} corresponds to executing uncoerced activities leading to fulfillment and satisfaction.
 
\section{Historical Development} %? Historic?

The emergence of free time is a result of a large set of historical developments. An awareness of free time cannot be developed without the awareness of time. Long before humans used tools to map time onto a scale, time was mostly perceived in a cyclic manner, regarded as a succession of recurring phases. The alteration of day and night, the intake of food, recurring rituals or seasonal phenomena are examples of the very beginning of an awareness of time. The mapping of time onto a linear scale, which allows to define a specific point in time as well as to measure the duration of activities, was introduced by the ancient Egyptians with the sothic cycle and the first sundials. \citep[p.25-27]{whitrow1989} The precision was coarse and varied in different regions of the world until centuries later mechanical clocks raised the precision to a practical level \citep[p.103]{whitrow1989}. However, it was primarily the reformation and the accompanying protestant work ethic that paved the way for an economization of time as part of European capitalism \citep[p.22]{weber2006}.

With the industrial revolution, factory workers had to align their division of time to the rhythm of machines. They lost the freedom of allocating autonomously their working-hours. \citep[p.160]{whitrow1989}. With industrialization a clear distinction of free and working time became possible. Basically, time is bound equally to each one of us and cannot run short. Money is the collectible counterpart and can be transferred to other individuals independently. As loan was now solely coupled to working hours, the evaluation of time with money transformed time into a scarce resource that each individual was able to sell to employers \citep[p.54]{marx1867}. 

The industrial revolution did rise the number of working hours to a maximum \citep[p.98]{prahl2002}. It also initiated a large set of ongoing processes to streamline and densify time in line with the maxim ``time is money'' - the famous statement by one of the founding fathers of the United States of America, Benjamin Franklin \citep[p.22]{weber2006}. The invention of the railroad introduced the standard time as it became necessary to have a unified time across cities. The telegraph and the undersea cable boosted the communication and shortened the physical distance between people. Shortly after, mass production of pocket watches started. Watches became ubiquitous and began to dominate everybody's life - Lewis Mumford said, ``the clock, not the steam-engine, is the key-machine of the modern industrial age''. \citep[p.161]{whitrow1989} The densification of time continued with the invention of electricity and the light bulb that leveraged the alteration of day and night. With the spreading of roll film cameras people were able to easily capture and fix every thinkable situation in time. At the beginning of the 20th century, the first mass production of cars (e.g. Ford Model T) and airplanes further decreased the physical distance and increased the traveling speed. \newline
Among the densification until the middle of the 20th century, the number of working hours decreased due to the efforts of the labor movement and social policies. It has since been recognized that the mass production and the resulting high supply of goods cannot persist without a respective demand. Consequently, the amount of free time increased. \cite[p.99-100]{prahl2002} With the advent of additional free time, the demand for a higher quality of free time increased and it's commercialization started.\citep[p.116]{scheuch1972} The growing wealth also increased the number of leisure facillities. Different flows of leisure oportunities greater occupied people's free time along all social classes: The mass media made it possible for workers to listen to operas of the high class. New music and dancing styles were developed. New kinds of art got published and even new life styles had been tried out. Free time gained a new quality. \cite[p.106]{prahl2002}

In the mid 20th century many countries finally had been established the five-day workweek\footnote{In Germany the trade union DGB titled ``Samstags gehört Vati mir'' (german for ``At saturdays dad is mine'').} and flexible working hours. In western countries free time gradually transformed into a time of mass consumption. The notion of performance in our working-life had been applied to each individuals free time, competing with consumption, status symbols, travel and leisure performance in general. \citep[p.112]{prahl2002}

In the end of the 20th century, the invention and the ongoing emergence of the world wide web initiated another intensive densification of time. The complexity of information that can be transferred at a time increased spectacuraly \citep[p.45]{wajcman2014}. The best breeding ground for a sheer expansion of the globalization. The distances of global trade had been almost made meaningless as people were able to send different kinds of information in real time \citep[p.17]{wajcman2014}. The competition intensified and time is even more money and a timeframe of an individual becomes more precious. Besides the rise of globalization, the information age also changed the way people communicate personally. The communcation via the world wide web became ubiquitos. The technology makes it possible to correspond to more people at a time, than ever before. In addition, messengers and mails allow to schedule replies whenever possible without consuming time in meetings and goodbyes. The society transformed itself into an ``always on'' culture that puts the focus on being always available. Responding late is not welcome. A time for a break is rare. The quality of communication decreases in favor of speed.

Technological progress is measured by time savings. Faster means of transportation\footnote{The Hyperloop One will need less than a hour to transport people from Melbournce to Sydney. The company says it is not ``seeling transportation, [it is] selling time''. \cite{hyperloop2017}} and communication dominate the news. Increasing the speed in transporation and communication is desireable. Unfortunately, people invest saved time in new activities and increase the activites per day instead of making room for leisure. This presses us in time.

\section{Contemporary Organization}

\begin{itemize} 
	\item Wie haben wir damals ratschlage bekommen?
	\item Macher zu wähler
\end{itemize} 

% Wenn wir fremde Ziele zu unseren machen, entsteht auf Dauer ungesunder Stress.
% Großer Stress entsteht, wenn man etwas macht, das einem nicht entspricht, wenn man mit Aufgaben konfrontiert ist, mit denen man sich nicht innerlich verbinden kann.

\chapter{The Machine}
This chapter presents the conception and implementation of a machine to let users plan their free time with fellow people more efficiently. It should be noted that maximizing leisure time cannot be considered as the primary goal. Somebodies leisure time finds its total maximun as soon as the individual is not able to accomplish neccessary tasks in the individual's free time anymore. For that reason, the goal of the machine is not to maximize leisure time. The machine shall extend one's leisure time that implies an overall optimization of free time. In the following a set of possible precautions will be presented in respect of developments in free time as presented in the previous chapter.

\section{Concept}
The application combines the advantages along the mentioned methods in contemporary organization with the assigned goal to reduce the time for planning events in favor of time used for realize events.



 % Autonomy of Decisions
% - *Allgemein:* Was soll der Ratgeber leisten? 
% - Sol sich nicht auf individuellen Rat verlassen.
% - Alle Beteiligten werden mit einbezogen.
% - Vorschläge geben und dem Nutzer die Entscheidung überlassen, ob er dieser nachgehen mag.
% - Besitzt nicht den Anspruch, dem Menschen die Entscheidungen abzunehmen. Auch deshalb, da die Aufgabe, Empfehlungen zu geben, nicht wohldefiniert ist, da diese zu vielschichtig ist.
% - Weiterhin bestehen geläufige Probleme der Empfehlungsalgorithmen, wie bspw. dem Cold Start und der Filterblase

%Die maximierung findet seine grenze an dem zeitunkt, wo notwendige aufgaben innherlab der freizeit nicht mehr erledigt werden können. Aus diesem grund kann auch nicht von der maximierung der leisure die rede sein, sondern mein ziel ist es, die freizeit zu optimieren und dabei die leisure auf ein erhöhtes maß zu bringen. ( Optimierung der freizeit, erhöhung und nicht maximierung der leisure. )

Abgrenzung zu Facebook: Thesen unterfüttern, auch ob Facebook die Zeit verplempert -> inwiefern limitiert und verhindert der Prototyp dies
\section{Implementation}

\subsection{Design Language}

\subsection{Groups of Interest and Events}

\subsection{Explorative Components}

\subsection{The Recommendation Algorithm}
\begin{itemize} 
	\item Wie sollen die Empfehlungen konkret gegeben werden? (Mit Codebeispielen)
	\item Andere Aktivitäten empfehlen, basierend auf bewertete Aktivitäten (explizit, kontextsensitiv)
\end{itemize} 
	
\section{Required Hardware and Software}

\chapter{Methodology}

\section{Technical Validation}
\begin{itemize} 
	\item Beispielhaftes Modell und Anwendung des Algorithmus.
\end{itemize} 

\section{User-centered Validation}

\subsection{Recruitement}

\subsection{Participants}

\subsection{System Usability Score}
\begin{itemize} 
	\item Evaluation der Usability
	\item Ausschließen, dass Empfehlungen bewertet werden und nicht die Defizite der Applikation
\end{itemize} 

\subsection{Opening Questionnaire}

\subsection{Closing Questionnaire}
\begin{itemize} 
	\item Nutzerzentriertes Evaluieren der Empfehlungsalgorithmen.
	\item Ist es gelungen, die Freizeitgestaltung zu vereinfachen?
\end{itemize} 

\section{Structured Interview}
\begin{itemize} 
	\item Hast du mind. einen Streifzug unternommen?
	\item Hat es deine Freizeitgestaltung direkt oder indirekt beeinflusst (hatte es impact)?
\end{itemize}

\chapter{Results}

\section{Technical Validation}

\section{System Usability Score}

\section{Opening Questionnaire}

\section{Closing Questionnaire}

\section{Structured Interview}

\chapter{Analysis}
Ergebnisse der Studie wiederholen und in den Kontext stellen

\chapter{Conclusion}
\begin{itemize} 
	\item Thema, wie in der Einführung, aufgreifen
	\item Ausblick formulieren
	\begin{itemize} 
		\item Implementation von Bucketlisten?
	\end{itemize} 
\end{itemize} 


\begin{appendix} 
\chapter{Appendix}
\newpage
\section{Studymaterial}
\vspace*{\fill}
%\center{\frame{\includegraphics[width=.98\textwidth, page=1]{apx/Studieninformationen}}}
\label{lab:Studymaterial}
\vspace*{\fill}
\end{appendix}

% Literaturverzeichnis
\clearpage
\pagenumbering{roman}
\bibliographystyle{mlu_ifg}
\bibliography{books}

%Abbildungsverzeichnis
\listoffigures

% Erklärung
\chapter*{Declaration of academic honesty}
\thispagestyle{empty}
XXXX
\begin{center}
\begin{tabular}{lp{2em}l} 
 \hspace{5cm}   && \hspace{4cm} \\\cline{1-1}\cline{3-3} 
 Bremen, \today    && Jan-Hendrik Wolf 
\end{tabular} 
\end{center}
\end{document}