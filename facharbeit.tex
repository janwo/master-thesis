\documentclass[12pt,numbers=noenddot,parskip,bibliography=totocnumbered,listof=totocnumbered]{scrreprt}
\usepackage[paper=a4paper,left=30mm,right=25mm,top=30mm,bottom=30mm]{geometry}
\usepackage[utf8]{inputenc}
\usepackage[hyphens]{url}
\usepackage[english]{babel}
\usepackage{tocstyle}
\usetocstyle{allwithdot}
\usepackage{lmodern}
\usepackage[numbers, sort&compress]{natbib}
\usepackage{graphicx} 
\usepackage{subcaption}
\usepackage{rotating}
\usepackage{transparent}
\usepackage{multirow}
\usepackage{tabularx}
\usepackage{pifont}
\usepackage{setspace}
\usepackage{amsmath}
\usepackage[automark]{scrlayer-scrpage}

% Stil der Seiten
\pagestyle{scrheadings}
\clearscrheadfoot

%Abstand der Fussnoten
\deffootnote{1em}{1em}{\textsuperscript{\thefootnotemark\ }}

%Zeilenabstand
\onehalfspacing

%Regeln, bis zu welcher Tiefe Überschriften angezeigt werden sollen (Anzeige der Überschriften im Verzeichnis / Anzeige der Nummerierung)
\setcounter{tocdepth}{3}
\setcounter{secnumdepth}{3}

%Kopfzeile (Kapitel / Standard)
\ohead{\transparent{0.5}\headmark\transparent{0.5}\transparent{1}}

%Fußzeile (Kapitel / Standard)
\ofoot[\rlap{\hspace{.5cm}\rule[-0.7ex]{.8pt}{\baselineskip}\hspace{.5cm}\pagemark}]{\rlap{\hspace{.5cm}\rule[-0.7ex]{.8pt}{\baselineskip}\hspace{.5cm}\pagemark}}

%Kapitel in Kopfzeile ohne Zahl
\renewcommand*{\chaptermarkformat}{}

%Schriftarten
\addtokomafont{pagenumber}{\sffamily \upshape}
\addtokomafont{pageheadfoot}{\sffamily \upshape}\usepackage[defaultfam,light,tabular,lining]{montserrat}
\usepackage[T1]{fontenc}
\renewcommand*\oldstylenums[1]{{\fontfamily{Montserrat-TOsF}\selectfont #1}}

\begin{document}
	
% TItelseite
\begin{titlepage}
\null
\vfill
% The Collective Guide: A machine for collective recommendations and its desire for disclosure
% Honestly transforms experiences in a collective into public recommendations
% The Honest Machine: Transforms experiences in a collective into honest recommendations
% The bold attempt of transforming experiences in a collective into  recommendations
% The Honest Machine: The attempt of converting experiences in a collective into honest recommendations

\Huge\textsf{\textbf{Free your free time
\vspace{0.5em}}}\\
\LARGE\textsf{ An exemplary machine to rebase digital benefits to revert the trend of social solitary } % A proposal how to fight ubiquitous accessibility with it's own weapons
\vspace{1.5em}\\
\Large\textsf{Jan-Hendrik Wolf}
\vfill
\vfill
\vfill
\small{Thesis submitted to the Universität Bremen in partial fulfillment of the requirements for a M.Sc. degree in Digital Media\\
Bremen, 30. Juli 2017}
\end{titlepage}

% Anfang des Bodies
\pagenumbering{roman}

%Inhaltsverzeichnis
\tableofcontents

\chapter*{Abstract}

\chapter{Introduction}
\pagenumbering{arabic}
Gone are the days of slow movement and solid free time. Gone are the days of accepting boredom and time of waiting. The influence of media changed the way we organise our free time. We stress ourselves, trying to fill any possible gap in our free time. Any intervening period during a bus ride, in the waiting room or alone at home is filled with a constantly growing abundance of minor activities. Maintain your friendships, social web profiles and all manner of incoming information. \\
Social networks like Facebook or search engines like Google opened up enough opportunities to choose from and made it unreasonable to feel boredom and vacancy, which was an acceptable part of your free time thirty years ago. Not staying in touch over a certain timeframe is not taken for granted anymore due to the growing number of channels in communication. The digitalisation made all this possible and the smartphone is the most popular toolbox to pursue this habits. \\
This development is continued by examples like shopping malls become larger, the number of sports grows and even ordinary commercial products are sold as a new experience. This confusing number of opportunities make us feel afraid to miss a thing. We are constantly trying to transform interims into productive free time, anyhow refuse ourself to the unexpected. According to a series of studies \citep{freizeitmonitor}, we are constantly increase our personal stress level this way. The artificial impression is build up that time is running short. 

This phenomenon has impact on the way we spend our free time. The winning activities for the last five years had been activities you are doing alone, whereas social activities had been the most notably loosers. It seems evident that it takes more time to meet and spend time with each other personally, than staying in contact via messengers and social media. However, why do we not use the advantages of computer technology in a way that motivates people to preferably spend their free time in a context of social activities again? Is it possible that an application, the embodiment of digitalisation in the current state, can invert the trend towards a free time with social activities in focus?

The thesis will call attention to the constant change of one's free time under decades of digital influence. It will investigate to what extend the ubiquitous accessibility of everyones attention certainly played it's part to this development and proposes a possible solution to invert the trend. A puristic application that does not focus on chat components, self-portraits and it's time-consuming relatives. The aim is to conceptualise and implement an application that helps to organise one's free time with a small timestamp and it's constant motivation to make people meet each other in person again rather than digitally. The thesis will end with an evaluation of application users to validate whether it is a rather drastic and humorously ironic idea to come up with another digital application or a step forward to one's free time with more social activities.

% single activity does not last  more than two hours. 
% most notably surfing the web, listening to music and go to the gym.
% Fülle an Ideen für Freizeitaktivitäten und Interessen.
% Menschen kennen meist nur einen kleinen Bereich mangels des Wissens der anderen Bereiche.
% Ratgeber sind nicht individuell auf den Einzelnen abgestimmt, parteiisch und konzentrieren sich auf spezielle Ereignisse.
% Seit XXXX gibt es Empfehlungsalgorithmen, werden zunehmend beliebter und es gibt eine Fülle an nützlichen Anwendungsbeispielen: Diese Algorithmen könnte man nutzen, um Nutzern bei der Filterung zu helfen und die Ideen und Interessen an einem Ort zu speichern.
%Bestehende, automatisierte Ratgeber (z.B. Facebook) geben undurchsichtige Vorschläge - es wird keine Auskunft über über deren Generierung gegeben. 
%Wieviele Quellen bezieht die Empfehlung bspw. ein? Mystifizierte Vorschläge.

\chapter{The Evolution Of Free Time}

\section{Definition}

\section{Historical Development}
\begin{itemize} 
	\item Wie haben wir damals ratschlage bekommen?
	\item Macher zu wähler
\end{itemize} 

\section{Contemporary Organization}

\chapter{The Machine}

\section{Concept}
% Autonomy of Decisions
% - *Allgemein:* Was soll der Ratgeber leisten? 
% - Sol sich nicht auf individuellen Rat verlassen.
% - Alle Beteiligten werden mit einbezogen.
% - Vorschläge geben und dem Nutzer die Entscheidung überlassen, ob er dieser nachgehen mag.
% - Besitzt nicht den Anspruch, dem Menschen die Entscheidungen abzunehmen. Auch deshalb, da die Aufgabe, Empfehlungen zu geben, nicht wohldefiniert ist, da diese zu vielschichtig ist.
% - Weiterhin bestehen geläufige Probleme der Empfehlungsalgorithmen, wie bspw. dem Cold Start und der Filterblase
\section{Implementation}

\subsection{Design Language}

\subsection{Groups of Interest and Events}

\subsection{Explorative Components}

\subsection{The Recommendation Algorithm}
\begin{itemize} 
	\item Wie sollen die Empfehlungen konkret gegeben werden? (Mit Codebeispielen)
	\item Andere Aktivitäten empfehlen, basierend auf bewertete Aktivitäten (explizit, kontextsensitiv)
\end{itemize} 
	
\section{Required Hardware and Software}

\chapter{Methodology}

\section{Technical Validation}
\begin{itemize} 
	\item Beispielhaftes Modell und Anwendung des Algorithmus.
\end{itemize} 

\section{User-centered Validation}

\subsection{Application testees}

\subsection{Opening Questionnaire}

\subsection{System Usability Score}
\begin{itemize} 
	\item Evaluation der Usability
	\item Ausschließen, dass Empfehlungen bewertet werden und nicht die Defizite der Applikation
\end{itemize} 

\subsection{Closing Questionnaire}
\begin{itemize} 
	\item Nutzerzentriertes Evaluieren der Empfehlungsalgorithmen.
	\item Ist es gelungen, die Freizeitgestaltung zu vereinfachen?
\end{itemize} 

\chapter{Results}

\section{Technical Validation}

\section{Opening Questionnaire}

\section{System Usability Score}

\section{Closing Questionnaire}

\chapter{Analysis}
Ergebnisse der Studie wiederholen und in den Kontext stellen

\chapter{Conclusion}
\begin{itemize} 
	\item Thema, wie in der Einführung, aufgreifen
	\item Ausblick formulieren
	\begin{itemize} 
		\item Implementation von Bucketlisten?
	\end{itemize} 
\end{itemize} 


\begin{appendix} 
\chapter{Appendix}
\newpage
\section{Studymaterial}
\vspace*{\fill}
%\center{\frame{\includegraphics[width=.98\textwidth, page=1]{apx/Studieninformationen}}}
\label{lab:Studymaterial}
\vspace*{\fill}
\end{appendix}

% Literaturverzeichnis
\clearpage
\pagenumbering{roman}
\bibliographystyle{mlu_ifg}
\bibliography{books}

%Abbildungsverzeichnis
\listoffigures

% Erklärung
\chapter*{Declaration of academic honesty}
\thispagestyle{empty}
XXXX
\begin{center}
\begin{tabular}{lp{2em}l} 
 \hspace{5cm}   && \hspace{4cm} \\\cline{1-1}\cline{3-3} 
 Bremen, \today    && Jan-Hendrik Wolf 
\end{tabular} 
\end{center}
\end{document}