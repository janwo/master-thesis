\documentclass[12pt,numbers=noenddot,parskip,bibliography=totocnumbered,listof=totocnumbered]{scrreprt}
\usepackage[paper=a4paper,left=30mm,right=25mm,top=30mm,bottom=30mm]{geometry}
\usepackage[utf8]{inputenc}
\usepackage[hyphens]{url}
\usepackage[english]{babel}
\usepackage{tocstyle}
\usetocstyle{allwithdot}
\usepackage{lmodern}
\usepackage[numbers, sort&compress]{natbib}
\usepackage{graphicx} 
\usepackage{subcaption}
\usepackage{rotating}
\usepackage{transparent}
\usepackage{multirow}
\usepackage{tabularx}
\usepackage{pifont}
\usepackage{setspace}
\usepackage{amsmath}
\usepackage[automark]{scrlayer-scrpage}

% Stil der Seiten
\pagestyle{scrheadings}
\clearscrheadfoot

%Abstand der Fussnoten
\deffootnote{1em}{1em}{\textsuperscript{\thefootnotemark\ }}

%Zeilenabstand
\onehalfspacing

%Regeln, bis zu welcher Tiefe Überschriften angezeigt werden sollen (Anzeige der Überschriften im Verzeichnis / Anzeige der Nummerierung)
\setcounter{tocdepth}{3}
\setcounter{secnumdepth}{3}

%Kopfzeile (Kapitel / Standard)
\ohead{\transparent{0.5}\headmark\transparent{0.5}\transparent{1}}

%Fußzeile (Kapitel / Standard)
\ofoot[\rlap{\hspace{.5cm}\rule[-0.7ex]{.8pt}{\baselineskip}\hspace{.5cm}\pagemark}]{\rlap{\hspace{.5cm}\rule[-0.7ex]{.8pt}{\baselineskip}\hspace{.5cm}\pagemark}}

%Kapitel in Kopfzeile ohne Zahl
\renewcommand*{\chaptermarkformat}{}

%Schriftarten
\addtokomafont{pagenumber}{\sffamily \upshape}
\addtokomafont{pageheadfoot}{\sffamily \upshape}\usepackage[defaultfam,light,tabular,lining]{montserrat}
\usepackage[T1]{fontenc}
\renewcommand*\oldstylenums[1]{{\fontfamily{Montserrat-TOsF}\selectfont #1}}

\begin{document}
	
% TItelseite
\begin{titlepage}
\null
\vfill
% The Collective Guide: A machine for collective recommendations and its desire for disclosure
% Honestly transforms experiences in a collective into public recommendations
% The Honest Machine: Transforms experiences in a collective into honest recommendations
% The bold attempt of transforming experiences in a collective into  recommendations

\Huge\textsf{\textbf{The Honest Machine
\vspace{0.5em}}}\\
\LARGE\textsf{ The attempt of converting experiences in a collective into honest recommendations }
\vspace{1.5em}\\
\Large\textsf{Jan-Hendrik Wolf}
\vfill
\vfill
\vfill
\small{Thesis submitted to the Universität Bremen in partial fulfillment of the requirements for a M.Sc. degree in Digital Media\\
Bremen, 30. Juli 2017}
\end{titlepage}

% Anfang des Bodies
\pagenumbering{roman}

%Inhaltsverzeichnis
\tableofcontents

\chapter*{Abstract}

\chapter{Introduction}
\pagenumbering{arabic}
Recommendations are an omnipresent phenomenon in our society. People ask their friends for being their shopping companion, buy magazines in order to know which product works best for them, commit their Sunday planning to their family members or scroll through explicitly selected content in the world wide web. We are constantly exposed to recommendations in conversations, commercials and in environments with subconscious influence. Companies are developing new types of recommendations with consistent improvements in a growing number of areas in the people's life. Recommendations made with machine learning can give better results than humans do, e.g. in cancer screening or fraud detection. As a reasonable result people start to fully accept recommendations made by machines in favour of human made recommendations \citep{mldoctors}. Notwithstanding of the fact that most of the recommendation systems do not expose the process and relevance of the decision-making to the end user. Thus, the end user has no grounds for comprehend or criticise made recommendations.

The thesis examines the development of recommendations from an unidirectional broadcast towards a high complex process by giving worth mentioning examples of recommendations made in media over the last decades. In what extend has the end user exchanged it's autonomy of decisions towards a higher quality of the recommendations. How can a recommendation system hold the balance between both aspects and how can such a application looks like in particular?

- Fülle an Ideen für Freizeitaktivitäten und Interessen.
- Menschen kennen meist nur einen kleinen Bereich mangels des Wissens der anderen Bereiche.
- Ratgeber sind nicht individuell auf den Einzelnen abgestimmt, parteiisch und konzentrieren sich auf spezielle Ereignisse.
- Seit XXXX gibt es Empfehlungsalgorithmen, werden zunehmend beliebter und es gibt eine Fülle an nützlichen Anwendungsbeispielen: Diese Algorithmen könnte man nutzen, um Nutzern bei der Filterung zu helfen und die Ideen und Interessen an einem Ort zu speichern.
- Bestehende, automatisierte Ratgeber (z.B. Facebook) geben undurchsichtige Vorschläge - es wird keine Auskunft über über deren Generierung gegeben. 
Wieviele Quellen bezieht die Empfehlung bspw. ein? Mystifizierte Vorschläge.
- "Lass den Ratgeber demokratisieren", jeden einzelnen Nutzer miteinbeziehen und Transparenz schaffen (z.B. auf Filterblase hinweisen).
- Ziel: Ansatz umsetzen, evaluieren und Erkenntnisse gewinnen, ob die Transparenz angenommen wird und die Planung der Freizeitgestaltung erleichtert wird.

\chapter{The Shine and Despair of Guides}

\section{Autonomy of Decisions}
- Wieso gibt es Ratgeber? Was suchen wir darin?
- Was geben wir ab? Was wollen wir entscheiden? Welchen Charakter haben die Erfahrungen?

historischen Überblick, der Heutiges mit einbezieht. Er sollte nicht nur faktisch sein, also das anführen, was es gab und gibt, sondern auch, wer das Publikum ist, wozu Rat gegeben wird
\section{The Guidebook}
Erläuterung der Vor- und Nachteile.

Stiftung Warentest

\section{The Human}
Erläuterung der Vor- und Nachteile.

Warum folgen wir dem Ratschlag unserees Freundes so gern, selbst wenn wir finden, dass er jetzt daneben gegriffen hatte. Sprich über einige konkrete und glaubhafte Fälle. Um so die Problematik zu kennzeichnen. Und gib dann drei automatische Fälle im Beispiel an.


Why do we ask for advice?


\section{The Machine}
Erläuterung der Vor- und Nachteile.

Auf solchem Wege kannst Du Dein Thema gewinnen. Das, was diese Systeme heute versuchen, ist vermutlich überhaupt nichts Neues. Zeige das auf! Im Gegenteil, es wird alter Wein in neuen Schläuchen sein. Was geht verloren, was wird gewonnen und wie wird ideologisch den Leuten vorgemacht, sie könnten jetzt sich viel besser verlassen auf das, was ihnen da gezeigt wird.  Sobald der Ratgeber irgendwie selbst kommerziell ist, glaube ich ihm kein Wort.

\chapter{The Honest Machine}

\section{The Principles}

- *Allgemein:* Was soll der Ratgeber leisten? 
- Sol sich nicht auf individuellen Rat verlassen.
- Alle Beteiligten werden mit einbezogen.
- Vorschläge geben und dem Nutzer die Entscheidung überlassen, ob er dieser nachgehen mag.
- Transparentsein beim Vorschlagen.
- Besitzt nicht den Anspruch, dem Menschen die Entscheidungen abzunehmen. Auch deshalb, da die Aufgabe, Empfehlungen zu geben, nicht wohldefiniert ist, da diese zu vielschichtig ist.
- Weiterhin bestehen geläufige Probleme der Empfehlungsalgorithmen, wie bspw. dem Cold Start und der Filterblase => Hinweise geben, wie z.B. "Ich schlage dir gerade etwas willkürliches vor".)
\section{The Execution}
\subsection{The recommendation algorithm}
- Wie sollen die Empfehlungen konkret gegeben werden? (Mit Codebeispielen)
	- Andere Aktivitäten empfehlen, basierend auf bewertete Aktivitäten (explizit, kontextsensitiv)
	- Andere Nutzer empfehlen, basierend auf bewertete Aktivitäten (implizit, kontextsensitiv)
\subsection{The application}
- Wie sollen die Empfehlungen präsentiert werden?
	- Aufbau der Applikation (Rudel, Events, Personen)
	- Aufbau der Empfehlungen (inkl. transparente Hinweise)
\section{The Evaluation}
- Evaluation der Usability (Ausschließen, dass Empfehlungen bewertet werden und nicht die Defizite der Applikation).
- Nutzerzentriertes Evaluieren der Empfehlungsalgorithmen.
- Beispielhaftes Modell und Anwendung des Algorithmus.
- Ist es gelungen, die Freizeitgestaltung zu vereinfachen?

\chapter{Conclusion}
Das Thema, wie in der Einführung, aufgreifen und die Ergebnisse der Studie wiederholen und in denKontext stellen. Anschließend einen Ausblick formulieren.

\begin{appendix} 
\chapter{Appendix}
\newpage
\section{Studymaterial}
\vspace*{\fill}
%\center{\frame{\includegraphics[width=.98\textwidth, page=1]{apx/Studieninformationen}}}
\label{lab:Studymaterial}
\vspace*{\fill}
\end{appendix}

% Literaturverzeichnis
\clearpage
\pagenumbering{roman}
\bibliographystyle{mlu_ifg}
\bibliography{books}

%Abbildungsverzeichnis
\listoffigures

% Erklärung
\chapter*{Declaration of academic honesty}
\thispagestyle{empty}
XXXX
\begin{center}
\begin{tabular}{lp{2em}l} 
 \hspace{5cm}   && \hspace{4cm} \\\cline{1-1}\cline{3-3} 
 Bremen, \today    && Jan-Hendrik Wolf 
\end{tabular} 
\end{center}
\end{document}