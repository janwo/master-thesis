\documentclass[12pt,numbers=noenddot,parskip,bibliography=totocnumbered,listof=totocnumbered]{scrreprt}
\usepackage[paper=a4paper,left=30mm,right=25mm,top=30mm,bottom=30mm]{geometry}
\usepackage[utf8]{inputenc}
\usepackage[hyphens]{url}
\usepackage[english]{babel}
\usepackage{tocstyle}
\usetocstyle{allwithdot}
\usepackage{lmodern}
\usepackage[numbers, sort&compress]{natbib}
\usepackage{graphicx} 
\usepackage{subcaption}
\usepackage{rotating}
\usepackage{transparent}
\usepackage{multirow}
\usepackage{tabularx}
\usepackage{pifont}
\usepackage{setspace}
\usepackage{amsmath}
\usepackage[automark]{scrlayer-scrpage}

% Stil der Seiten
\pagestyle{scrheadings}
\clearscrheadfoot

%Abstand der Fussnoten
\deffootnote{1em}{1em}{\textsuperscript{\thefootnotemark\ }}

%Zeilenabstand
\onehalfspacing

%Regeln, bis zu welcher Tiefe Überschriften angezeigt werden sollen (Anzeige der Überschriften im Verzeichnis / Anzeige der Nummerierung)
\setcounter{tocdepth}{3}
\setcounter{secnumdepth}{3}

%Kopfzeile (Kapitel / Standard)
\ohead{\transparent{0.5}\headmark\transparent{0.5}\transparent{1}}

%Fußzeile (Kapitel / Standard)
\ofoot[\rlap{\hspace{.5cm}\rule[-0.7ex]{.8pt}{\baselineskip}\hspace{.5cm}\pagemark}]{\rlap{\hspace{.5cm}\rule[-0.7ex]{.8pt}{\baselineskip}\hspace{.5cm}\pagemark}}

%Kapitel in Kopfzeile ohne Zahl
\renewcommand*{\chaptermarkformat}{}

%Schriftarten
\addtokomafont{pagenumber}{\sffamily \upshape}
\addtokomafont{pageheadfoot}{\sffamily \upshape}\usepackage[defaultfam,light,tabular,lining]{montserrat}
\usepackage[T1]{fontenc}
\renewcommand*\oldstylenums[1]{{\fontfamily{Montserrat-TOsF}\selectfont #1}}

\begin{document}
	
% TItelseite
\begin{titlepage}
\null
\vfill
% The Collective Guide: A machine for collective recommendations and its desire for disclosure
% Honestly transforms experiences in a collective into public recommendations
% The Honest Machine: Transforms experiences in a collective into honest recommendations
% The bold attempt of transforming experiences in a collective into  recommendations
% The Honest Machine: The attempt of converting experiences in a collective into honest recommendations

\Huge\textsf{\textbf{The freetime machine
\vspace{0.5em}}}\\
\LARGE\textsf{ The attempt of converting experiences in a collective into honest recommendations }
\vspace{1.5em}\\
\Large\textsf{Jan-Hendrik Wolf}
\vfill
\vfill
\vfill
\small{Thesis submitted to the Universität Bremen in partial fulfillment of the requirements for a M.Sc. degree in Digital Media\\
Bremen, 30. Juli 2017}
\end{titlepage}

% Anfang des Bodies
\pagenumbering{roman}

%Inhaltsverzeichnis
\tableofcontents

\chapter*{Abstract}

\chapter{Introduction}
\pagenumbering{arabic}
Gone are the days of free and slow movement. Gone are the days of accepting boredom and time of waiting. The influence of media changed the way we organise our free time. We stress ourselves, trying to fill any possible gap in our free time. Any intervening period during a bus ride, in the waiting room or alone at home is filled with a constantly growing abundance of minor activities. Maintain your friendships, social web profiles and all manner of incoming information. \\
Social networks like Facebook or search engines like Google opened up enough opportunities to choose from and made it unreasonable to feel boredom and vacancy, which was an acceptable part of your free time thirty years ago. Not staying in touch over a certain timeframe is not taken for granted anymore due to the growing number of channels in communication. The digitalisation made all this possible and the smartphone is the most popular toolbox to pursue this habits. \\
This development is continued by examples like shopping malls become larger, the number of sports grows and even ordinary commercial products are sold as a new experience. This confusing number of opportunities make us feel afraid to miss a thing, we are constantly trying to transform interims and things into active free time. According to a series of studies \citep{freizeitmonitor}, we are constantly increase our personal stress level this way. The artificial impression is build up that time is running short. 

This phenomenon has impact on the way we spend our free time. The winning activities for the last five years had been activities you are doing alone, whereas social activities had been the most notably loosers. It seems evident that it takes more time to meet and spend time with each other personally, than staying in contact via messengers and social media. However, why do we not use the advantages of computer technology in a way that motivates people to preferably spend their free time in a context of social activities again? Is it possible that an application, the embodiment of digitalisation in the current state, can invert the trend towards a free time with social activities in focus?

The thesis is going to call attention to the constant change of one's free time under decades of digital influence. It will investigate to what extend the ubiquitous accessibility of everyones attention certainly played it's part to this development and proposes a possible solution to invert the trend. And leaves it up to the reader wheter  a drastic and humorously ironic idea to propose an solution that comes with another digital application?  concept that 




single activity does not last  more than two hours. 
% most notably surfing the web, listening to music and go to the gym.

\newpage

- Fülle an Ideen für Freizeitaktivitäten und Interessen.
- Menschen kennen meist nur einen kleinen Bereich mangels des Wissens der anderen Bereiche.
- Ratgeber sind nicht individuell auf den Einzelnen abgestimmt, parteiisch und konzentrieren sich auf spezielle Ereignisse.
- Seit XXXX gibt es Empfehlungsalgorithmen, werden zunehmend beliebter und es gibt eine Fülle an nützlichen Anwendungsbeispielen: Diese Algorithmen könnte man nutzen, um Nutzern bei der Filterung zu helfen und die Ideen und Interessen an einem Ort zu speichern.
- Bestehende, automatisierte Ratgeber (z.B. Facebook) geben undurchsichtige Vorschläge - es wird keine Auskunft über über deren Generierung gegeben. 
Wieviele Quellen bezieht die Empfehlung bspw. ein? Mystifizierte Vorschläge.
- "Lass den Ratgeber demokratisieren", jeden einzelnen Nutzer miteinbeziehen und Transparenz schaffen (z.B. auf Filterblase hinweisen).
- Ziel: Ansatz umsetzen, evaluieren und Erkenntnisse gewinnen, ob die Transparenz angenommen wird und die Planung der Freizeitgestaltung erleichtert wird.

\chapter{The Shine and Despair of Guides}

\section{Autonomy of Decisions}
- Wieso gibt es Ratgeber? Was suchen wir darin?
- Was geben wir ab? Was wollen wir entscheiden? Welchen Charakter haben die Erfahrungen?

historischen Überblick, der Heutiges mit einbezieht. Er sollte nicht nur faktisch sein, also das anführen, was es gab und gibt, sondern auch, wer das Publikum ist, wozu Rat gegeben wird
\section{The Guidebook}
Erläuterung der Vor- und Nachteile.

Stiftung Warentest

\section{The Human}
Erläuterung der Vor- und Nachteile.

Warum folgen wir dem Ratschlag unserees Freundes so gern, selbst wenn wir finden, dass er jetzt daneben gegriffen hatte. Sprich über einige konkrete und glaubhafte Fälle. Um so die Problematik zu kennzeichnen. Und gib dann drei automatische Fälle im Beispiel an.


Why do we ask for advice?


\section{The Machine}
Erläuterung der Vor- und Nachteile.

Auf solchem Wege kannst Du Dein Thema gewinnen. Das, was diese Systeme heute versuchen, ist vermutlich überhaupt nichts Neues. Zeige das auf! Im Gegenteil, es wird alter Wein in neuen Schläuchen sein. Was geht verloren, was wird gewonnen und wie wird ideologisch den Leuten vorgemacht, sie könnten jetzt sich viel besser verlassen auf das, was ihnen da gezeigt wird.  Sobald der Ratgeber irgendwie selbst kommerziell ist, glaube ich ihm kein Wort.

\chapter{The Honest Machine}

\section{The Principles}

- *Allgemein:* Was soll der Ratgeber leisten? 
- Sol sich nicht auf individuellen Rat verlassen.
- Alle Beteiligten werden mit einbezogen.
- Vorschläge geben und dem Nutzer die Entscheidung überlassen, ob er dieser nachgehen mag.
- Transparentsein beim Vorschlagen.
- Besitzt nicht den Anspruch, dem Menschen die Entscheidungen abzunehmen. Auch deshalb, da die Aufgabe, Empfehlungen zu geben, nicht wohldefiniert ist, da diese zu vielschichtig ist.
- Weiterhin bestehen geläufige Probleme der Empfehlungsalgorithmen, wie bspw. dem Cold Start und der Filterblase => Hinweise geben, wie z.B. "Ich schlage dir gerade etwas willkürliches vor".)
\section{The Execution}
\subsection{The recommendation algorithm}
- Wie sollen die Empfehlungen konkret gegeben werden? (Mit Codebeispielen)
	- Andere Aktivitäten empfehlen, basierend auf bewertete Aktivitäten (explizit, kontextsensitiv)
	- Andere Nutzer empfehlen, basierend auf bewertete Aktivitäten (implizit, kontextsensitiv)
\subsection{The application}
- Wie sollen die Empfehlungen präsentiert werden?
	- Aufbau der Applikation (Rudel, Events, Personen)
	- Aufbau der Empfehlungen (inkl. transparente Hinweise)
\section{The Evaluation}
- Evaluation der Usability (Ausschließen, dass Empfehlungen bewertet werden und nicht die Defizite der Applikation).
- Nutzerzentriertes Evaluieren der Empfehlungsalgorithmen.
- Beispielhaftes Modell und Anwendung des Algorithmus.
- Ist es gelungen, die Freizeitgestaltung zu vereinfachen?

\chapter{Conclusion}
Das Thema, wie in der Einführung, aufgreifen und die Ergebnisse der Studie wiederholen und in denKontext stellen. Anschließend einen Ausblick formulieren.

OUTLOOK

BUcketlisten

\begin{appendix} 
\chapter{Appendix}
\newpage
\section{Studymaterial}
\vspace*{\fill}
%\center{\frame{\includegraphics[width=.98\textwidth, page=1]{apx/Studieninformationen}}}
\label{lab:Studymaterial}
\vspace*{\fill}
\end{appendix}

% Literaturverzeichnis
\clearpage
\pagenumbering{roman}
\bibliographystyle{mlu_ifg}
\bibliography{books}

%Abbildungsverzeichnis
\listoffigures

% Erklärung
\chapter*{Declaration of academic honesty}
\thispagestyle{empty}
XXXX
\begin{center}
\begin{tabular}{lp{2em}l} 
 \hspace{5cm}   && \hspace{4cm} \\\cline{1-1}\cline{3-3} 
 Bremen, \today    && Jan-Hendrik Wolf 
\end{tabular} 
\end{center}
\end{document}