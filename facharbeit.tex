\documentclass[12pt,numbers=noenddot,parskip,bibliography=totocnumbered,listof=totocnumbered]{scrreprt}
\usepackage[paper=a4paper,left=30mm,right=25mm,top=30mm,bottom=30mm]{geometry}
\usepackage[utf8]{inputenc}
\usepackage[hyphens]{url}
\usepackage[english]{babel}
\usepackage{tocstyle}
\usetocstyle{allwithdot}
\usepackage{lmodern}
\usepackage[numbers, sort&compress]{natbib}
\usepackage{graphicx} 
\usepackage{subcaption}
\usepackage{filecontents}
\usepackage{rotating}
\usepackage{multirow}
\usepackage{tabularx}
\usepackage{pifont}
\usepackage{setspace}
\usepackage{amsmath}
\usepackage[automark]{scrlayer-scrpage}

\begin{filecontents}{books.bib}

@MISC{winrichtlinien,
author = "{Microsoft Corporation}",
year = "2014",
title = "Windows App Development: Guidelines for targeting",
howpublished = "\url{http://msdn.microsoft.com/en-us/library/windows/apps/hh465326.aspx}",
note = "Zuletzt aufgerufen am 06.08.2014"
}

@MISC{rproject,
author = "{The R Foundation}",
year = "2014",
title = "The R Project for Statistical Computing",
howpublished = "\url{http://www.r-project.org/}",
note = "Zuletzt aufgerufen am 22.09.2014"
}

@MISC{applerichtlinien,
author = "{Apple Inc.}",
year = "2014",
title = "iOS Human Interface Guidelines: Layout",
howpublished = "\url{https://developer.apple.com/library/ios/documentation/UserExperience/Conceptual/MobileHIG/LayoutandAppearance.html}",
note = "Zuletzt aufgerufen am 09.08.2014"
}

@MISC{androidrichtlinien,
author = "{{Google Inc.}}",
year = "2014",
title = "Android Developers: Metrics and Grids",
howpublished = "\url{http://developer.android.com/design/style/metrics-grids.html}",
note = "Zuletzt aufgerufen am 09.08.2014"
}

@MISC{sonyverge,
author = "Dante D'Orazio",
year = "2012",
title = "Sony Xperia Sola's 'floating touch' screen technology explained",
howpublished = "\url{http://www.theverge.com/2012/3/14/2871193/sony-xperia-sola-floating-touch-hover-event-screen-technology}",
note = "Zuletzt aufgerufen am 07.08.2014"
}

@ARTICLE{monkeys,
author = "K. Dandekar and B.I. Raju and M.A. Srinivasan",
title = "3-D finite-element models of human and monkey fingertips to investigate the mechanics of tactile sense",
pages = "S. 682-691",
year = "2003",
journal = "Journal of biomechanical Engineering",
volume = "125"
}

@INCOLLECTION{jin,
author = "Zhao Xia Jin and Tom Plocher and Liana Kiff",
title = "Touch screen user interfaces for older adults: button size and spacing",
booktitle = "Proceedings of the 4th international conference on Universal access in human computer interaction: coping with diversity (UAHCI'07)",
publisher = "Springer Verlag",
year = "2007",
editor = "Constantine Stephanidis",
pages = "933-941",
address = "Berlin, Heidelberg"
}

@ARTICLE{touchmouse,
author = "Charlotte Travis and Pietro Murano",
title = "A comparative study of the usability of touch-based and mouse-based interaction",
journal = "Int. J. Pervasive Computing and Communications",
volume = "10",
number = "1",
year = "2014",
pages = "115-134"
}

@MISC{dorol,
author = "{Doro Deutschland}",
year = "2014",
title = "Doro Liberto 810",
howpublished = "\url{http://www.dorodeutschland.de/Produkte/Mobiltelefone-und-Zubehor/Doro-Liberto-810/}",
note = "Zuletzt aufgerufen am 24.08.2014"
}

@MISC{gOne,
author = "Andreas Brohme",
year = "2014",
title = "Android One: Google startet seine Billig-Smartphone-Offensive",
howpublished = "\url{http://www.spiegel.de/netzwelt/gadgets/android-one-google-soll-billig-smartphone-bald-praesentieren-a-989376.html}",
note = "Zuletzt aufgerufen am 04.09.2014"
}

@INPROCEEDINGS{mousetouchmk,
author = "Farzan Sasangohar and I. Scott MacKenzie and Stacey D. Scott",
title = "Evaluation of mouse and touch input for a tabletop display using Fitts' reciprocal tapping task",
booktitle = "Proceedings of the Human Factors and Ergonomics Society Annual Meeting",
volume = "53",
number = "12",
pages = "839-843",
year = "2009", 
organization = "SAGE Publications"
}

@MISC{synaptics,
author = "{Synaptics Inc.}",
year = "1998",
title = "Synaptics TouchPad Interfacing Guide (510-000080-A, Second Edition)",
howpublished = "\url{http://www.synaptics.com/sites/default/files/ACF126.pdf}",
note = "Zuletzt aufgerufen am 18.08.2014"
}

@BOOK{nui,
author = "Daniel Wigdor and Dennis Wixon",
title = "Brave NUI World: Designing Natural User Interfaces for Touch and Gesture",
year = "2011",
edition = "1st",
publisher = "Morgan Kaufmann Publishers Inc.",
address = "San Francisco, CA, USA",
}

@INCOLLECTION{sus,
address = "London",
author = "John Brooke",
booktitle = "Usability evaluation in industry",
editor = "Jordan, P. W. and Weerdmeester, B. and Thomas, A. and Mclelland, I. L.",
publisher = "Taylor and Francis",
title = "SUS: A quick and dirty usability scale",
year = "1996"
}

@ARTICLE{sus2013,
title = "SUS: A Retrospective",
author = "John Brooke",
journal = "Journal of Usability Studies",
volume="8",
number="2",
pages="29-40",
year="2013"
}

@ARTICLE{fitts,
author = "I. Scott MacKenzie",
title = "Fitts' Law As a Research and Design Tool in Human-computer Interaction",
journal = "Hum.-Comput. Interact.",
issue_date = "March 1992",
volume = "7",
number = "1",
year = "1992",
pages = "91-139",
numpages = "49",
publisher = "L. Erlbaum Associates Inc.",
address = "Hillsdale, NJ, USA",
} 
\end{filecontents}

% Stil der Seiten
\pagestyle{scrheadings}
\clearscrheadfoot

%Abstand der Fussnoten
\deffootnote{1em}{1em}{\textsuperscript{\thefootnotemark\ }}

%Zeilenabstand
\onehalfspacing

%Regeln, bis zu welcher Tiefe Überschriften angezeigt werden sollen (Anzeige der Überschriften im Verzeichnis / Anzeige der Nummerierung)
\setcounter{tocdepth}{3}
\setcounter{secnumdepth}{3}

%Seitenstil: Kopfzeile ohne Nummer und normale Fußzeile, z.B. für Nicht-Kapitel-Seiten im Inhaltsverzeichnis. Vererbung mit Präfix=[Seitenstil].
\newpairofpagestyles{nochapternumber}{\ohead{\textbf{\headmark}}\ofoot[\rlap{\hspace{.5cm}\rule[-0.7ex]{.8pt}{\baselineskip}\hspace{.5cm}\pagemark}]{\rlap{\hspace{.5cm}\rule[-0.7ex]{.8pt}{\baselineskip}\hspace{.5cm}\pagemark}}
}

%Kopfzeile (Kapitel / Standard)
\ohead{\textbf{\headmark}\rlap{\hspace{.5cm}\rule[-0.7ex]{.8pt}{\baselineskip}\hspace{.5cm}\thechapter}}

%Fußzeile (Kapitel / Standard)
\ofoot[\rlap{\hspace{.5cm}\rule[-0.7ex]{.8pt}{\baselineskip}\hspace{.5cm}\pagemark}]{\rlap{\hspace{.5cm}\rule[-0.7ex]{.8pt}{\baselineskip}\hspace{.5cm}\pagemark}}

%Kapitel in Kopfzeile ohne Zahl
\renewcommand*{\chaptermarkformat}{}

%Schriftarten
\addtokomafont{pagenumber}{\sffamily \upshape}
\addtokomafont{pageheadfoot}{\sffamily \upshape}\usepackage[defaultfam,light,tabular,lining]{montserrat} %% Option 'defaultfam'
%% only if the base font of the document is to be sans serig
\usepackage[T1]{fontenc}
\renewcommand*\oldstylenums[1]{{\fontfamily{Montserrat-TOsF}\selectfont #1}}

%-------------------
%Ende des Kopfbereiches
%-------------------

%START---------------------------------------------------------------------------------
\begin{document}
%Beginn der Titelseite
\begin{titlepage}
\null
\vfill
%#The Collective Guide: A machine for collective recommendations and its desire for disclosure
\Huge\textsf{\textbf{The Collective Guide
\vspace{0.5em}}}\\
\LARGE\textsf{A machine for disclosed / transparent recommendations in a collective}
\vspace{1.5em}\\
\Large\textsf{Jan-Hendrik Wolf}
\vfill
\vfill
\vfill
\small{Thesis submitted to the Universität Bremen in partial fulfillment of the requirements for a M.Sc. degree in Digital Media\\
Bremen, 30. Juli 2017}

%Sommersemester 2017
\end{titlepage}
%Ende der Titelseite

%Inhaltsverzeichnis
\tableofcontents
\thispagestyle{nochapternumber}

%START-----------------------------------------------------------------------------------

% Kein Umbruch: \mbox{Xperia™ sola}  \mbox{$7$ mm}

%  (siehe Abbildung~\ref{samsungeinleitung1})

%\begin{figure}
%\centering
%\subcaption{Normalzustand der Galerie.}{\includegraphics[width=0.49\textwidth]{img/software1_nohover.png}}
%\hfill
%\subcaption{Bildvergrößerung bei Annäherung.}{\includegraphics[width=0.49\textwidth]{img/software1_hover.png}}
%\caption{Objektvergrößerungen bei Annäherung des Fingers mit aktiviertem \mbox{AirView™} auf einem \mbox{Samsung Galaxy S4}.}
%\label{samsungeinleitung1}
%\end{figure}

% \citep{touchmouse}

%\begin{table}
%\centering
%\renewcommand{\arraystretch}{2}
%\setlength{\tabcolsep}{12pt}
%\begin{tabular}{ l c c c }
%Plattform & Hersteller & Minimale Größe & Empfohlene Größe\\\hline
%Android \citep{androidrichtlinien} & Google Inc. & $7$ mm & $9$ mm\\
%iOS \citep{applerichtlinien} & Apple Inc. & $7$ mm & $7$ mm\\
%Windows \citep{winrichtlinien} & Microsoft Corp. & $7$ mm & $9$ mm\\
%\end{tabular}
%\caption{Mindestgrößen von Bedienelementen für Geräte mit berührungsempfindlichen Bildschirmen.}
%\label{leitfadenalle}
%\end{table}

%(med: $23$ Jahre, \O: $25{,}6$ Jahre)

% \mbox{1920$\times$1080}


%\begin{figure}
%\centering
%\subcaption{Phase 1: Halten der grünen Schaltfläche.}%{\includegraphics[width=0.64\textwidth]{img/test2_phase1.jpg}}
%\vfill
%\subcaption{Phase 2: Blaue Schaltfläche leuchtet pseudozufällig auf}]%{\includegraphics[width=0.64\textwidth]{img/test2_phase2.jpg}}
%\vfill
%\subcaption{Phase 3: Berühren der blauen Schaltfläche.}%{\includegraphics[width=0.64\textwidth]{img/test2_phase3.jpg}}
%\caption{Aufbau von Test 2: Drei unterschiedliche Phasen pro Durchgang.}
%\label{test2phasen}
%\end{figure}

%{\ttfamily Annäherungserkennung} mit den nominalen Zuständen \emph{ausgeschaltet} und \emph{eingeschaltet}.

\chapter*{Abstract}

\chapter{Introduction}
\pagenumbering{arabic}


- Fülle an Ideen für Freizeitaktivitäten und Interessen.
- Menschen kennen meist nur einen kleinen Bereich mangels des Wissens der anderen Bereiche.
- Ratgeber sind nicht individuell auf den Einzelnen abgestimmt, parteiisch und konzentrieren sich auf spezielle Ereignisse.
- Seit XXXX gibt es Empfehlungsalgorithmen, werden zunehmend beliebter und es gibt eine Fülle an nützlichen Anwendungsbeispielen: Diese Algorithmen könnte man nutzen, um Nutzern bei der Filterung zu helfen und die Ideen und Interessen an einem Ort zu speichern.
- Bestehende, automatisierte Ratgeber (z.B. Facebook) geben undurchsichtige Vorschläge - es wird keine Auskunft über über deren Generierung gegeben. Wieviele Quellen bezieht die Empfehlung bspw. ein? => Mystifizierte Vorschläge.
- "Lass uns den Ratgeber demokratisieren", jeden einzelnen Nutzer miteinbeziehen und Transparenz schaffen (z.B. auf Filterblase hinweisen).
- Ziel: Ansatz umsetzen, evaluieren und Erkenntnisse gewinnen, ob die Transparenz angenommen wird und die Planung der Freizeitgestaltung erleichtert wird.



Warum folgen wir dem Ratschlag unserees Freundes so gern, selbst wenn wir finden, dass er jetzt daneben gegriffen hatte. Sprich über einige konkrete und glaubhafte Fälle. Um so die Problematik zu kennzeichnen. Und gib dann drei automatische Fälle im Beispiel an.

\chapter{The Shine and Despair of Guides}

\section{Autonomy of Decisions}
- Wieso gibt es Ratgeber? Was suchen wir darin?
- Was geben wir ab? Was wollen wir entscheiden? Welchen Charakter haben die Erfahrungen?

historischen Überblick, der Heutiges mit einbezieht. Er sollte nicht nur faktisch sein, also das anführen, was es gab und gibt, sondern auch, wer das Publikum ist, wozu Rat gegeben wird
\section{The Guidebook}
Erläuterung der Vor- und Nachteile.

Stiftung Warentest

\section{The Human}
Erläuterung der Vor- und Nachteile.

\section{The Machine}
Erläuterung der Vor- und Nachteile.

Auf solchem Wege kannst Du Dein Thema gewinnen. Das, was diese Systeme heute versuchen, ist vermutlich überhaupt nichts Neues. Zeige das auf! Im Gegenteil, es wird alter Wein in neuen Schläuchen sein. Was geht verloren, was wird gewonnen und wie wird ideologisch den Leuten vorgemacht, sie könnten jetzt sich viel besser verlassen auf das, was ihnen da gezeigt wird.  Sobald der Ratgeber irgendwie selbst kommerziell ist, glaube ich ihm kein Wort.

\chapter{The Collective Guide}

\section{The Principles}

- *Allgemein:* Was soll der Ratgeber leisten? 
- Sol sich nicht auf individuellen Rat verlassen.
- Alle Beteiligten werden mit einbezogen.
- Vorschläge geben und dem Nutzer die Entscheidung überlassen, ob er dieser nachgehen mag.
- Transparentsein beim Vorschlagen.
- Besitzt nicht den Anspruch, dem Menschen die Entscheidungen abzunehmen. Auch deshalb, da die Aufgabe, Empfehlungen zu geben, nicht wohldefiniert ist, da diese zu vielschichtig ist.
- Weiterhin bestehen geläufige Probleme der Empfehlungsalgorithmen, wie bspw. dem Cold Start und der Filterblase => Hinweise geben, wie z.B. "Ich schlage dir gerade etwas willkürliches vor".)
\section{The Execution}
\subsection{The recommendation algorithm}
- Wie sollen die Empfehlungen konkret gegeben werden? (Mit Codebeispielen)
	- Andere Aktivitäten empfehlen, basierend auf bewertete Aktivitäten (explizit, kontextsensitiv)
	- Andere Nutzer empfehlen, basierend auf bewertete Aktivitäten (implizit, kontextsensitiv)
\subsection{The application}
- Wie sollen die Empfehlungen präsentiert werden?
	- Aufbau der Applikation (Rudel, Events, Personen)
	- Aufbau der Empfehlungen (inkl. transparente Hinweise)
\section{The Evaluation}
- Evaluation der Usability (Ausschließen, dass Empfehlungen bewertet werden und nicht die Defizite der Applikation).
- Nutzerzentriertes Evaluieren der Empfehlungsalgorithmen.
- Beispielhaftes Modell und Anwendung des Algorithmus.
- Ist es gelungen, die Freizeitgestaltung zu vereinfachen?

\chapter{Conclusion}
Das Thema, wie in der Einführung, aufgreifen und die Ergebnisse der Studie wiederholen und in denKontext stellen. Anschließend einen Ausblick formulieren.

\begin{appendix} 
\chapter{Appendix}
\newpage
\section{Studymaterial}
\vspace*{\fill}
%\center{\frame{\includegraphics[width=.98\textwidth, page=1]{apx/Studieninformationen}}}
\label{lab:Studymaterial}
\vspace*{\fill}
\end{appendix}

%ENDE-----------------------------------------------------------------------------------
% Literaturverzeichnis
\clearpage
\pagenumbering{roman}
\bibliographystyle{mlu_ifg}
\bibliography{books}

%Abbildungsverzeichnis
\listoffigures

% Erklärung
\chapter*{Declaration of academic honesty}
\thispagestyle{empty}
XXXX
\begin{center}
\begin{tabular}{lp{2em}l} 
 \hspace{5cm}   && \hspace{4cm} \\\cline{1-1}\cline{3-3} 
 Bremen, \today    && Jan-Hendrik Wolf 
\end{tabular} 
\end{center}
\end{document}